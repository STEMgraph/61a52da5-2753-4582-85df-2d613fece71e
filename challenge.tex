\begin{challenge}
    \chatitle{Using xpdf for displaying PDF files}
    \begin{chadescription}
    When the widespread use of graphical terminals became more common in the early 1990s, problems started to appear. Documents that someone created on a specific computer, weren't necessarily portable to other platforms. Portable in this context means, that a layout that the author intended, just wouldn't be displayed properly on somebody else's computer. As long as people were just using \texttt{.txt} files on their text-based terminals, there was no problem. But as the possibility to include picture into documents and display them on a graphical shell offered more customization options, customization problems also grew. It was Adobe, that came up with the idea of a portable file format, that could represent written documents properly - independent of which computer was used to display it. Hence the name: Portable Document Format or short \textbf{PDF}. PDF offered three major improvements to generate consistent renderings, while still keeping the file-size from exploding.
    \begin{enumerate}
        \item including Graphics and Fonts into the file-format itself
        \item providing a reader-software for multiple platforms - Adobe Acrobat
        \item protecting documents from manipulation by a non-privileged reader
    \end{enumerate}
    Adobe Acrobat was used to generate and read PDFs and was sold by Adobe for multiple platforms. One of the reasons by the way why they used C++ to write it. When Adobe realized, that it would help their market penetration to have everybody being able to read they released a cost-free spin-off called \textit{Adobe Acrobat Reader}. In this challenge you shall find out, how to use a free PDF reader to display documents like these.
    \end{chadescription}
    \begin{task}
    Open your terminal and check whether your \texttt{DISPLAY} Variable is set correctly.
    \begin{questions}
        \item Which command can you use on your system to display all environment-variables?
        \item Which command can you use to read all env-variables, redirect the result into a filtering-tool and only display the wanted variables?
        \item Which command can you use to only display the wanted environment-variable?
        \item What is the content of your \texttt{DISPLAY} variable?
    \end{questions}
    If your display-variable is not set correctly, go back to the \href{https://www.github.com/STEMgraph/}{first X-Server Challenge}.
    \end{task}
    \begin{task}
    At this point, you should already have installed the x11-apps toolkit to test your connection. Use \texttt{xeyes} to test the connection to your x11-Server.
    \begin{questions}
        \item Which command can you use to start \texttt{xeyes}?
        \item Explain in your own words, what happens when a user-program wants to display something using the X11 protocol!
    \end{questions}
    If this didn't work as expected for you, go back to the \href{https://www.github.com/STEMgraph/}{first X-Server Challenge}.
    \end{task}
    \begin{task}
    Use your preferred package manager to install the \texttt{xpdf} package. If you want to build xpdf from source, follow the Link to the \href{https://gitlab.com/xpdf-mirror/xpdf}{xpdf} repository.
    \begin{questions}
        \item Which command can you use to install the \texttt{xpdf} package?
        \item In which programming language was xpdf written?
        \item Can you figure out, which GUI Library xpdf uses?
        \item Run \texttt{xpdf --version}, what is the output?
    \end{questions}
    \end{task}
    \begin{task}
    \texttt{xpdf} can display PDF files. Run \texttt{xpdf <filename>} to display the PDF file \textit{<filename>}. If you don't have a PDF file, you can download a simple one from \href{https://constitutioncenter.org/media/files/constitution.pdf}{https://constitutioncenter.org/media/files/constitution.pdf}. To download a file into your \texttt{\textasciitilde} directory, use \texttt{wget <URL>}.
    \begin{questions}
        \item Run the command \texttt{xpdf -z 8}, what is the output?
        \item Run the command \texttt{xpdf -z 50}; then run the command \texttt{xpdf -z 150}, what is the output?
        \item Run the command \texttt{xpdf -rv}, what is the output?
    \end{questions}
    \end{task}
    \begin{task}
    While installing xpdf, you also installed the \texttt{poppler-utils} package. Part of this package is the \texttt{pdftotext} command.
    \begin{questions}
        \item What does the command \texttt{pdftotext <filename>} do?
    \end{questions}
    \end{task}
    \end{challenge}
    